\chapter{Conclusions and Future Work}

\section{Conclusion}

The main aim of this project was to create a means to encourage people to walk more and help people discover more about the area around them. In that sense, based on the feedback received from users who tested and used the application, we have achieved what we set out to achieve.

A number of methods were researched and employed in the project in an attempt to motivate the user to exercise more, including gamification and exercising in groups. Gamification was implemented in the application by allowing achievements to be gained for walking further and more regularly, while group exercising was implemented by encouraging users to walk with one another using walk invitations. 

Back-end development, including Javascript and the Node.js environment, were completely new to me before the start of the project and therefore required a lot more time to learn than other parts of the project. Following research into the types of technologies that needed to be used in the back-end, as well as a few tutorials to help with API design and authentication, I was able to create an elegant and well structured API using Javascript that served the mobile application.

Throughout the course of this project, we have undertaken the following:

\begin{enumerate}
  \item Researched existing fitness applications and technologies to determine which features are most popular and what technologies would be most useful to me (Chapter \ref{chapter:background}).
  \item Designed and implemented a connected, fully-functioning back-end Node.js API and a front-end iOS mobile app that allows a user to record their walks, view points of interest near to them, gain achievements to increase their score and invite one another to go on a walk (Chapters \ref{chapter:design} \& \ref{chapter:implementation}).
  \item Tested each component using automated testing services and evaluated the project with real users to gauge their opinion on design, ease-of-use and how useful the application is (Chapter \ref{chapter:evaluation}).
\end{enumerate}

We have produced an application that provides a number of innovative features relating to promoting exercise and discovering new places, including viewing local points of interest and inviting users to go on a walk. Although some of the features need polishing and the app does still have its limitations, what I have created provides a good stepping stone to create a complete social walking app.

\section{Future Work} \label{section:future-work}

Listed below are some of the future extensions that I would like to implement in the application given more time. Some of the extensions are based on current features that can be extended and others are new ideas that I thought of throughout the project.

\subsection{Live walk tracking system} \label{subsection:live-walk-tracking-system}

Due to time constraints, the proposed feature to invite users on a walk was not fully implemented. The current system only allows the walk to be tracked by the invitation sender, while the recipients cannot view the current walk until it has completed and is visible on their profile.

The proposed extension plans to show all the users of a shared walk the walk tracking view at the same time, so that all users are able to view walk statistics and discover points of interest around them. To make sure that all users were viewing the same map at the same time, data would need to be streamed from the invitation sender to all of the recipients.

\subsection{Add photos to walk}

While the current implementation only allowing for a walk route and its statistics to be saved, a step forward would be allowing the user to take photos of points of interest along their walk. These photos would then appear on the map once the walk had been saved.

\subsection{Popular and nearby walks}

A new section of the app could be created that shows both popular and nearby walks created by other users. Walks will therefore obviously need to be able to be rated, possibly using a like/dislike system. Pre-made sets of well-known walks could also be shown in this section -- the London Loop being a good example of one.

\subsection{Favourite points of interest}

After using the application a lot, I realised there were points of interest viewed during a walk that I wanted to refer back to later. The favourites system will enable the user to save a number of their favourite or interesting places while tracking a walk. These favourites will then be visible in a new section of the app even after the walk has finished.

\subsection{Activity feed}

While users can invite each other to walk with one another, there is no current system to connect or view each other's profile. This could be done by allowing users to add each other as friends. Another section of the app could then be used to display a recent activity feed to show users what walks their friends have recorded recently.


