\chapter{Evaluation}

Evaluation must be conducted throughout the project to assess both the technical quantitative aspects of the project so that the project functions correctly and the qualitative aspects, such as the design and ease of use of the produced application.

Additional evaluation is then performed near the completion of the project in order to assess how well the project's objectives were achieved and whether the project can be considered a success.

%The evaluation can be split into two main categories:
%
%\begin{itemize}
%  \item \textbf{Quantitative evaluation:} assessing how well 
%\end{itemize}

\section{Software Validation}

Throughout the project, I ensured that each feature of the application was well tested to minimise bugs that could arise later on in the project. When implementing a feature in the project, the back-end element was normally completed and tested first to guarantee that when the front-end implementation was taking place I would be working with a fully-working element from the back-end. Testing was performed using the continuous integration tool Travis CI chosen in Section \ref{subsection:version-control}. This meant that tests were run automatically when commits were made to Git, allowing me to see at exactly what point any tests failed.

\subsection{API Testing}

The purpose of API testing is to ensure that all endpoints function correctly and handle any erroneous input as expected. Mocha **ref*** was the main testing framework used to test the API. It supports asynchronous callbacks, which is necessary for making calls to the API in test cases. Each test case is identified by a unique string to differentiate between each case and recognise which test cases have failed.

To make the actual API requests in a test case, the Supertest ***ref*** framework, allowing for assertions of HTTP requests. The structure of one of the test cases can be seen in Listing \ref{listing:api-test}. The \verb|describe()| and \verb|it()| functions specify the structure of the tests and uniquely identify each test case respectively, with the latter providing a \verb|done| function (line 3) to be called once the test case has finished executing its request. The HTTP request is made on line 4, where \verb|request| is a reference to the Supertest dependency, specifying the HTTP method and path (line 5). Subsequent functions are then called on this request to make assertions on elements of the response using the \verb|expect()| function (lines 6-11). In the example shown, these assertions include checking the response is of JSON type, checking that the HTTP response code is \textbf{200 OK} and checking the number of users returned is zero (since this particular request searches for a user with an invalid

\medskip

\begin{listing}
  \centering
  \begin{lstlisting}[language=javascript]
describe('GET /search/:userInfo', function() {
  describe('Valid user search', function () {
    it("should return empty array with name not found", function(done) {
      request(app)
        .get(uriPrefix + '/search/invalid name')
        .expect('Content-Type', /json/)
        .expect(function(res) {
          res.body.success.should.be.equal(true);
          res.body.users.should.have.length(0);
        })
        .expect(200, done);
    });
  });
});
  \end{lstlisting}
  \caption{Structure of a test case for the API}
  \label{listing:api-test}
\end{listing}

%The file that these test cases are written in is recognised by the project package to be the main test file for the project, meaning this test file is executed when the  \verb|test| command is run.  

The file that these test cases are written in is listed in the project's configuration file, so that this test file is automatically executed when the default \verb|test| command is run. When commits are made to Git, Travis CI then automatically runs the the \verb|test| command, indicating which test cases passed. If a test case failed, I was able to pinpoint the exact part of the endpoint that was affected due to the the unique names I gave to each test case. This was invaluable in speeding up the error fixing process.

\subsection{Mobile App Testing}

Testing the front-end mobile app was not performed in as much detail as the back-end, mainly because of forgetfulness and concentration on the implementation at hand. With that being said, many parts of the logic and UI of the application were tested thoroughly. To create a behaviour-driven development testing environment and provide English-like assertions by using the Quick ***ref*** and Nimble ***ref*** frameworks respectively, so as to match the way in which the API was tested.

Calls to the API were tested to ensure that responses were handled properly in \verb|APIManager|. In order to prevent unwanted requests to the API during testing, some HTTP requests were stubbed using the Mockingjay framework ***ref*** and mocked responses were returned to the app instead. For example, when making a valid request to register a new user, the actual API request is not made to prevent a superfluous record from being created in the database. In this case, a mocked response is returned containing a \textbf{200 OK} status code and the emulated successful response that would have been returned from the server. For invalid requests, no database records are ever created and so HTTP requests do not need to be mocked -- instead the actual error response returned from the API can be used in a test case.

To make sure that different components within the app worked together correctly, integration tests (or UI tests as they are known in iOS) were written.

% write more about integration tests

\section{User Testing}

I aimed to provide users with versions of the application as early as possible and often throughout the project so as to gain feedback frequently and iterate the application multiple times throughout the project based on this feedback. The feedback that was received from users ranged from bug fixes in the application to subjective views on features of the app that could be improved or changed.

\subsection{Bug Fixes}

The following section lists some of the bugs that were found in my code based on feedback from testing my application in the real world, either by myself or from other users that I provided the app to. For each error that was found, the reason that this error occurred and the solution that fixed the error is also listed.


\noindent \textbf{Error:} app crashes making any network request with no internet connection.

\begin{itemize}
  \item \textbf{Reason:} every time a network request is made in \verb|APIManager|, the status code of the response is obtained to see if an error has been returned. However, if there is no internet connection, there is no response from the request and so the status code is empty. This results in the Swift equivalent of a null pointer error, causing the app to crash.
  
  \item \textbf{Solution:} the status code of the response is only used if the response itself is not null, otherwise a default network error is returned.
\end{itemize}
  
\noindent \textbf{Error:} app crashes when creating an account after scrolling the text fields off screen.

\begin{itemize}
  \item \textbf{Reason:} table view cells in iOS are reusable, meaning the cells and their associated data that are not on the screen are not in memory and are recreated when they reappear on the screen. When pressing the button to create an account, the data the user entered from the text fields in the table view cells is retrieved and passed to the \verb|register()| function in \verb|APIManager|. If one or more of the cells is not on the screen, its data does not exist due to its reusability and so a null pointer error occurs when trying to obtain the value passed into \verb|register()|.
  
  \item \textbf{Solution:} instead of retrieving the data from cells when pressing the register button, a global class array was used to store the data from the cells as the user is entering it. This means that even if some cells are not on the screen when registering, the data is still stored previously and can be passed to the \verb|register()| function without any danger of any parameters being null.
\end{itemize}
  
\noindent \textbf{Error:} when tracking a particularly long walk, the image produced contains rendering issues where the map route is blurred, as shown in Figure ***.

\begin{itemize}
  \item \textbf{Reason:} the walk route image was produced by rendering an image of the visible map view on the screen using Core Graphics -- a framework used in iOS to perform 2D drawing. To do this, the map view needs to be zoomed out so that the entire walk route is fit in the view. In doing so, the polyline used to draw the walk route on the map, discussed in Section \ref{section:tracking-walks}, did not always render properly for reasons that I was not able to discover.
  
  \item \textbf{Solution:} the \verb|MKMapSnapShotter| class, part of the MapKit framework, was used to render an image of a map rather than using Core Graphics. By specifying a coordinate region and various other options, an image is rendered of that area of a map. The map route still had to be rendered separately afterwards using Core Graphics, but I found this was the best solution at the time to avoid the rendering issues presented by the previous method.
\end{itemize}

\subsection{Final Survey}

% make qualitative evaluation into quantitive
% survey on ease of use, design, rating of features, etc.

When the implementation phase of the project was completed, I conducted a survey about the app as a way to quantify the qualitative aspects of the project that needed to be evaluated. These qualitative aspects include the ease of use of the application, its design and how useful each feature is to the user. The full results of this survey can be seen in Appendix ***, with a summary of results discussed in the following section.

\section{Objective Reflection}

% gauge if project is success

One way to gauge whether the project can be considered a success is to reflect back on the broad objectives proposed in Section \ref{section:objectives}. The objectives are listed below, each with an explanation as to whether it was achieved as a result of the quantitative and qualitative evaluation discussed in the previous two sections.

\begin{enumerate}[label=\textbf{Obj \arabic*}]
  \item \textbf{Encourage walking:} the basic features aimed to achieve this objective, namely inviting users to go on a walk and gamification, were implemented.
  \item \textbf{Help discover the world:}
  \item \textbf{Test and evaluate with real users:}
\end{enumerate}

\subsection{Project Timeline}

To reflect on how well I managed my time during the project I compared the original project plan proposed at the start of the project to the actual time taken to complete each task. The comparison between the two can be seen in Table \ref{table:project-timeline-comparison}.

I had planned to implement a great deal of the project, including setting up the back-end and skeleton app, before the exams at the end of the second term on \nth{20} March so as to provide myself with a platform to continue the project after this date. In reality however, due to the combination of the workload I had during this time and the steep learning curve for implementing the back-end, I did not implement as much as I had hoped.

Furthermore, I did not anticipate how long the initial phase of the API implementation would take me. I had no experience in Javascript or any back-end development before the start of the project and so there was a much steeper learning curve to that of the front-end development. On the other hand, some of the front-end features actually took a shorter amount of time than what I had planned. For example, inviting users to go on a walk actually only took around two weeks to implement rather than the three weeks I had outlined in Table \ref{table:project-timeline-comparison}, meaning that I gained some time back from the time lost implementing the API setup.

I had made sure to allow for extra time at the end of the implementation phase so that the problems listed above did not hinder the project and I was able to complete my intended features in time, albeit not having enough time to implement the extensions I had planned before the project. These were formulated into future extensions, which are listed in detail in Section \ref{section:future-work}.

\begin{table}[hbt]
  \centering
  \begin{tabular}{|l|| *{16}{c|}}
    \hline
    \multicolumn{17}{|c|}{\textsc{Proposed}}\\
    \hline
    \multirow{2}{*}{\textbf{Activity}} & \multicolumn{3}{c|}{\textbf{February}} & \multicolumn{4}{c|}{\textbf{March}} & \multicolumn{4}{c|}{\textbf{April}} & \multicolumn{5}{c|}{\textbf{May}}\\
    \cline{2-17}
    & 13 & 20 & 27 & 6 & 13 & 20 & 27 & 3 & 10 & 17 & 24 & 1 & 8 & 15 & 22 & 29\\
    \hline
    Skeleton app & \cellcolor{BrickRed} &&&& \multirow{10}{*}{\rotatebox[origin=c]{90}{\textls{REVISION}}} & \multirow{10}{*}{\rotatebox[origin=c]{90}{\textls{EXAMS}}} &&&&&&&&&&\\
    \hhline{*{5}{-}~~*{10}{-}}
    Set up web server &\multicolumn{2}{c|}{\cellcolor{BrickRed}}&&&&&&&&&&&&&&\\
    \hhline{*{5}{-}~~*{10}{-}}
    Set up database &\multicolumn{2}{c|}{\cellcolor{BrickRed}}&&&&&&&&&&&&&&\\
    \hhline{*{5}{-}~~*{10}{-}}
    Login system &&\multicolumn{2}{c|}{\cellcolor{BrickRed}}&&&&&&&&&&&&&\\
    \hhline{*{5}{-}~~*{10}{-}}
    Track walks &&&\multicolumn{2}{c|}{\cellcolor{BrickRed}}&&&&&&&&&&&&\\
    \hhline{*{5}{-}~~*{10}{-}}
    User Profile &&&&&&&\multicolumn{2}{c|}{\cellcolor{BrickRed}}&&&&&&&&\\
    \hhline{*{5}{-}~~*{10}{-}}
    Popular walks &&&&&&&&\multicolumn{2}{c|}{\cellcolor{BrickRed}}&&&&&&&\\
    \hhline{*{5}{-}~~*{10}{-}}
    Invite users for walk &&&&&&&&&&\multicolumn{3}{c|}{\cellcolor{BrickRed}}&&&&\\
    \hhline{*{5}{-}~~*{10}{-}}
    Gamification &&&&&&&&&&&&&\multicolumn{2}{c|}{\cellcolor{BrickRed}}&&\\
    \hhline{*{5}{-}~~*{10}{-}}
    Extensions &&&&&&&&&&&&&&&\multicolumn{2}{c|}{\cellcolor{BrickRed}}\\
    \hline
    \hline
    \multicolumn{17}{|c|}{\textsc{Actual}}\\
    \hline
    \multirow{2}{*}{\textbf{Activity}} & \multicolumn{3}{c|}{\textbf{February}} & \multicolumn{4}{c|}{\textbf{March}} & \multicolumn{4}{c|}{\textbf{April}} & \multicolumn{5}{c|}{\textbf{May}}\\
    \cline{2-17}
    & 13 & 20 & 27 & 6 & 13 & 20 & 27 & 3 & 10 & 17 & 24 & 1 & 8 & 15 & 22 & 29\\
    \hline
    Skeleton app & \cellcolor{OliveGreen} &&&& \multirow{10}{*}{\rotatebox[origin=c]{90}{\textls{REVISION}}} & \multirow{10}{*}{\rotatebox[origin=c]{90}{\textls{EXAMS}}} &&&&&&&&&&\\
    \hhline{*{5}{-}~~*{10}{-}}
    Set up web server &\multicolumn{2}{c|}{\cellcolor{OliveGreen}}&&&&&&&&&&&&&&\\
    \hhline{*{5}{-}~~*{10}{-}}
    Set up database &\multicolumn{2}{c|}{\cellcolor{OliveGreen}}&&&&&&&&&&&&&&\\
    \hhline{*{5}{-}~~*{10}{-}}
    Login system &&\multicolumn{2}{c|}{\cellcolor{OliveGreen}}&&&&&&&&&&&&&\\
    \hhline{*{5}{-}~~*{10}{-}}
    Track walks &&&\multicolumn{2}{c|}{\cellcolor{OliveGreen}}&&&&&&&&&&&&\\
    \hhline{*{5}{-}~~*{10}{-}}
    User Profile &&&&&&&\multicolumn{2}{c|}{\cellcolor{OliveGreen}}&&&&&&&&\\
    \hhline{*{5}{-}~~*{10}{-}}
    Popular walks &&&&&&&&\multicolumn{2}{c|}{\cellcolor{OliveGreen}}&&&&&&&\\
    \hhline{*{5}{-}~~*{10}{-}}
    Invite users for walk &&&&&&&&&&\multicolumn{3}{c|}{\cellcolor{OliveGreen}}&&&&\\
    \hhline{*{5}{-}~~*{10}{-}}
    Gamification &&&&&&&&&&&&&\multicolumn{2}{c|}{\cellcolor{OliveGreen}}&&\\
    \hhline{*{5}{-}~~*{10}{-}}
    Extensions &&&&&&&&&&&&&&&\multicolumn{2}{c|}{\cellcolor{OliveGreen}}\\
    \hline
  \end{tabular}
  \caption{Comparison between the proposed schedule for the project (top) and the actual schedule (bottom).}
  \label{table:project-timeline-comparison}
\end{table}

\section{Summary}

