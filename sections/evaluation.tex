\chapter{Evaluation}

Evaluation must be conducted throughout the project to assess both the technical quantitative aspects of the project so that the project functions correctly and the qualitative aspects, such as the design and ease of use of the produced application.

Additional evaluation is then performed near the completion of the project in order to assess how well the project's objectives were achieved and whether the project can be considered a success.

\section{Software Validation}

Throughout the project, I ensured that each feature of the application was well tested to minimise bugs that could arise later on in the project. When implementing a feature in the project, the back-end element was normally completed and tested first to guarantee that when the front-end implementation was taking place I would be working with a fully-working element from the back-end. Testing was performed using the continuous integration tool Travis CI chosen in Section \ref{subsection:version-control}. This meant that tests were run automatically when commits were made to Git, allowing me to see at exactly what point any tests failed.

\subsection{API Testing}

The purpose of API testing is to ensure that all endpoints function correctly and handle any erroneous input as expected. Mocha **ref*** was the main testing framework used to test the API. It supports asynchronous callbacks, which is necessary for making calls to the API in test cases. Each test case is identified by a unique string to differentiate between each case and recognise which test cases have failed.

To make the actual API requests in a test case, the Supertest ***ref*** framework, allowing for assertions of HTTP requests. The structure of one of the test cases can be seen in Listing \ref{listing:api-test}.

% explain listing
% add line numbers

\begin{listing}
  \centering
  \begin{lstlisting}[style=json]
describe('GET /search/:userInfo', function() {
  describe('Valid user search', function () {
    it('should return empty array with name not found', function(done) {
      request(app)
        .get(uriPrefix + '/search/invalid name')
        .expect('Content-Type', /json/)
        .expect(function(res) {
          res.body.success.should.be.equal(true);
          res.body.users.should.have.length(0);
        })
        .expect(200, done);
    });
  });
});
  \end{lstlisting}
  \caption{Structure of a test case for the API}
  \label{listing:api-test}
\end{listing}

%The file that these test cases are written in is recognised by the project package to be the main test file for the project, meaning this test file is executed when the  \verb|test| command is run.  

The file that these test cases are written in is listed in the project's configuration file, so that this test file is automatically executed when the default \verb|test| command is run. When commits are made to Git, Travis CI then automatically runs the the \verb|test| command, indicating which test cases passed. If a test case failed, I was able to pinpoint the exact part of the endpoint that was affected due to the the unique names I gave to each test case. This was invaluable in speeding up the error fixing process.

\subsection{Mobile App Testing}

Testing the front-end mobile app was not performed as much

\section{User Testing}

I aimed to provide users with versions of the application early and often throughout the project so as to gain feedback frequently and iterate the application multiple times throughout the project based on this feedback. The feedback that was received from users ranged from bug fixes in the application to subjective views on features of the app that could be improved or changed. The 

\section{Objective Reflection}

% gauge if project is success

One way to gauge whether the project can be considered a success is to reflect back on the broad objectives proposed in Section \ref{section:objectives}. The objectives are listed below, each with an explanation as to whether it was achieved as a result of the project:

\begin{enumerate}[label=\textbf{Obj \arabic*}]
  \item \textbf{Encourage walking:}
  \item \textbf{Help discover the world:}
  \item \textbf{Test and evaluate with real users:}
\end{enumerate}

\subsection{Project Timeline}

% compare project plan to actual plan

To reflect on how well I managed my time during the project I compared the original project plan proposed 

\begin{table}[hbt]
  \centering
  \begin{tabular}{|l|| *{16}{c|}}
    \hline
    \multicolumn{17}{|c|}{\textsc{Proposed}}\\
    \hline
    \multirow{2}{*}{\textbf{Activity}} & \multicolumn{3}{c|}{\textbf{February}} & \multicolumn{4}{c|}{\textbf{March}} & \multicolumn{4}{c|}{\textbf{April}} & \multicolumn{5}{c|}{\textbf{May}}\\
    \cline{2-17}
    & 13 & 20 & 27 & 6 & 13 & 20 & 27 & 3 & 10 & 17 & 24 & 1 & 8 & 15 & 22 & 29\\
    \hline
    \hline
    Skeleton app & \cellcolor{BrickRed} &&&& \multirow{10}{*}{\rotatebox[origin=c]{90}{\textls{REVISION}}} & \multirow{10}{*}{\rotatebox[origin=c]{90}{\textls{EXAMS}}} &&&&&&&&&&\\
    \hhline{*{5}{-}~~*{10}{-}}
    Set up web server &\multicolumn{2}{c|}{\cellcolor{BrickRed}}&&&&&&&&&&&&&&\\
    \hhline{*{5}{-}~~*{10}{-}}
    Set up database &\multicolumn{2}{c|}{\cellcolor{BrickRed}}&&&&&&&&&&&&&&\\
    \hhline{*{5}{-}~~*{10}{-}}
    Login system &&\multicolumn{2}{c|}{\cellcolor{BrickRed}}&&&&&&&&&&&&&\\
    \hhline{*{5}{-}~~*{10}{-}}
    Track walks &&&\multicolumn{2}{c|}{\cellcolor{BrickRed}}&&&&&&&&&&&&\\
    \hhline{*{5}{-}~~*{10}{-}}
    User Profile &&&&&&&\multicolumn{2}{c|}{\cellcolor{BrickRed}}&&&&&&&&\\
    \hhline{*{5}{-}~~*{10}{-}}
    Popular walks &&&&&&&&\multicolumn{2}{c|}{\cellcolor{BrickRed}}&&&&&&&\\
    \hhline{*{5}{-}~~*{10}{-}}
    Invite users for walk &&&&&&&&&&\multicolumn{3}{c|}{\cellcolor{BrickRed}}&&&&\\
    \hhline{*{5}{-}~~*{10}{-}}
    Gamification &&&&&&&&&&&&&\multicolumn{2}{c|}{\cellcolor{BrickRed}}&&\\
    \hhline{*{5}{-}~~*{10}{-}}
    Extensions &&&&&&&&&&&&&&&\multicolumn{2}{c|}{\cellcolor{BrickRed}}\\
    \hline
    \hline
    \multicolumn{17}{|c|}{\textsc{Actual}}\\
    \hline
    \multirow{2}{*}{\textbf{Activity}} & \multicolumn{3}{c|}{\textbf{February}} & \multicolumn{4}{c|}{\textbf{March}} & \multicolumn{4}{c|}{\textbf{April}} & \multicolumn{5}{c|}{\textbf{May}}\\
    \cline{2-17}
    & 13 & 20 & 27 & 6 & 13 & 20 & 27 & 3 & 10 & 17 & 24 & 1 & 8 & 15 & 22 & 29\\
    \hline
    \hline
    Skeleton app & \cellcolor{OliveGreen} &&&& \multirow{10}{*}{\rotatebox[origin=c]{90}{\textls{REVISION}}} & \multirow{10}{*}{\rotatebox[origin=c]{90}{\textls{EXAMS}}} &&&&&&&&&&\\
    \hhline{*{5}{-}~~*{10}{-}}
    Set up web server &\multicolumn{2}{c|}{\cellcolor{OliveGreen}}&&&&&&&&&&&&&&\\
    \hhline{*{5}{-}~~*{10}{-}}
    Set up database &\multicolumn{2}{c|}{\cellcolor{OliveGreen}}&&&&&&&&&&&&&&\\
    \hhline{*{5}{-}~~*{10}{-}}
    Login system &&\multicolumn{2}{c|}{\cellcolor{OliveGreen}}&&&&&&&&&&&&&\\
    \hhline{*{5}{-}~~*{10}{-}}
    Track walks &&&\multicolumn{2}{c|}{\cellcolor{OliveGreen}}&&&&&&&&&&&&\\
    \hhline{*{5}{-}~~*{10}{-}}
    User Profile &&&&&&&\multicolumn{2}{c|}{\cellcolor{OliveGreen}}&&&&&&&&\\
    \hhline{*{5}{-}~~*{10}{-}}
    Popular walks &&&&&&&&\multicolumn{2}{c|}{\cellcolor{OliveGreen}}&&&&&&&\\
    \hhline{*{5}{-}~~*{10}{-}}
    Invite users for walk &&&&&&&&&&\multicolumn{3}{c|}{\cellcolor{OliveGreen}}&&&&\\
    \hhline{*{5}{-}~~*{10}{-}}
    Gamification &&&&&&&&&&&&&\multicolumn{2}{c|}{\cellcolor{OliveGreen}}&&\\
    \hhline{*{5}{-}~~*{10}{-}}
    Extensions &&&&&&&&&&&&&&&\multicolumn{2}{c|}{\cellcolor{OliveGreen}}\\
    \hline
  \end{tabular}
  \caption{Comparison between the proposed schedule for the project (top) and the actual schedule (bottom).}
  \label{table:implementation-plan}
\end{table}

\section{Summary}

