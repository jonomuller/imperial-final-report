\chapter{Introduction}

\section{Motivation}

% Background - what is the problem?
% Why is the problem difficult to solve
% The idea which this project is solving

In this day and age, it is extremely important to keep fit. With the rapid development in technology in the recent years, people are more inclined to stay inside looking at a screen rather than to go outside and exercise. The main demographic seeing an increase in the use of smartphones and computers are teenagers and young men. The increased day-to-day use of technology is shown to have an impact on the number of obese adolescents in the UK \cite{Kautiainen2005}, which can increase the chances of people developing serious illnesses including diabetes \cite{Lazar2005}. It is therefore of great significance to provide a means of exercising that helps people, especially adolescents, keep healthy.

The difficulty in trying to encourage people to exercise lies in the ability to engage the user and find something that they enjoy to do. If exercising is seen as something that you enjoy, it then becomes a pleasure to do rather than a burden. Gamification plays an important part in helping people exercise, and has been used in lots of fitness apps available on the iOS App Store already \cite{Lister2014}. The range in which companies have implemented gamification into their fitness applications ranges from a score-based system where users can compete against their friends to a complete game that allows the user to explore the world around them. An example of the latter is the popular mobile game Pok\'{e}mon Go, where you have to walk around and capture virtual `creatures' that are scattered around in the real world.

Another problem in trying to keep fit is the difficulty of motivating yourself to exercise regularly. A study conducted in 2001 found that exercising with another person helped to reduce stress and increase calmness compared with exercising alone, however it also resulted in people being more tired \cite{Plante2001a}. As well as this, exercising with another person also allows you to motivate and set goals for each other to achieve, which can be a challenge when exercising by yourself.

The idea behind this project is to encourage people to walk more often via helping them to find out more about the area around them. Exploring your surroundings can be very interesting and walking is one of the best ways to discover new areas. Let's say, for example, that there was a monument along your commuting route. When travelling via another means of transport, such as a car or train, you might not have the chance to notice this monument. When walking, along with a tool which displays points of interest as you were walking, you would be able to see something that you may never have noticed before.

The idea also extends to helping people walk together to promote regular exercise. A way to do this is to allow the user to schedule walks for a point in the future, and invite their friends along to join them. This means that users will have a fixed event in their calendar which will help keep a more structured fitness routine.


\section{Objectives} \label{section:objectives}

The aim of this project is to produce a working application that encourages people to walk more and helps discover new places in the world. The main objectives for the project to measure success on are as follows:

\begin{enumerate}[label=\textbf{Obj \arabic*}]
  \item \textbf{Encourage walking:} users should be encouraged to keep fit and exercise more.  One way to do this is to make it easy for users to exercise with another person as it has been shown to increase motivation and reduce stress. Gamification is another method that could be used to encourage walking more. Users should be able to compete with their friends as a means of motivating one another.
  \item \textbf{Help discover the world:} while walking users should be able to discover new places or points of interest in the area around them, which will hopefully motivate them to explore new areas of the world as well as increasing their fitness at the same time.
  \item \textbf{Test and evaluate with real users:} evaluation should be conducted with real users to see if the project has an impact on how often they exercise. The full plan on how the project is going to be evaluated can be seen in Section \ref{chapter:evaluation-plan}.
\end{enumerate}

% add evaluation of project as objective, refer to evaluation plan section






