\chapter{Design}

\section{Overview}

The project can be split into two main sections -- the front-end mobile application and the back-end server to store data and communicate with the mobile app.

These two sections link together very closely and are both required to produce a working implementation. It is therefore extremely important to not only consider the design of the user interface but also more technical aspects, such as the way the user's data is stored in the database and how the API will communicate with the mobile app.

\section{Technology Choices}

The first part of the design of the project is to consider which technologies to use. The technologies that I chose stemmed either from the background research I conducted in the previous section or previous personal knowledge.

%\subsection{Mobile Operating System}

% move section in background to here?

\subsection{Third-party APIs}

As discussed in Section \ref{subsection:background-apis}, Apple Maps very well integrated with the Core Location framework in iOS, which makes it easy to deal with displaying the user's location on a map. It also provides convenient functions for coordinate conversions and calculating distances between two coordinates, which is useful for this project when trying to calculate the distance of a tracked walk.

For these reasons, I chose to use Apple Maps and their MapKit framework as the source of maps within the application. The one issue that I previously discussed with using this framework over Google Maps was the lack of detail in some areas of the maps. This factor was not deemed important enough to impact my decision and did not outweigh the benefits of the MapKit framework, as listed above.

% points of interest APIs

\subsection{Server Architecture}

Before deciding what technologies to use for the server, such as which programming language to use, we must consider the type of architectural style to use. The aim of the API exposed by the server is to provide an abstraction between complex database queries and provide functionality that a client can use to query and update details. The Representational state transfer (REST) architectural principle fits the needs for aim very well. Every component in a RESTful web service is a resource that can be accessed via the standard HTTP methods such as \textit{GET}, \textit{POST} and \textit{DELETE}. In comparison with other web services, such as the Simple Object Access Protocol (SOAP), RESTful services have much less overhead when sending data due to the extra XML header information sent when using SOAP. 

% language
% experience in python
% javascript, node js - good for this use because...

RESTful web services support the use of multiple different data formats to serialise responses from the server, including HTML, XML and JSON. I chose to use JSON as the data type because of its easy parsing in Javascript.

% talk about objects in javascript and similarity to json

\subsection{Database}

\subsection{API Deployment}

\subsection{File Storage}


\section{User Interface Design}

% design of app
% how screens are organised
% survey

\section{API Design}

% REST model, specific endpoints, etc.

\subsection{Database Design}

% types of models – user, walk, invite, etc.